\chapter{Data Reduction}
\label{chap:data_reduction}

\section{Open and view a dataset}


\section{Go full-automatic: Reduction for dummies}
  For good quality data the program supports a fully automatized mode, where all reduction parameters are automatically calculated.
  This mode will be applied automatically when more than one dataset is selected at the File Open Dialog.
  The direct beam measurement have to have lower numbers than the actual measurements or need to be set in advance for this method to work.
  
  The automatic algorithm performs the same steps as described in section \ref{sec:quick_start}, while trying to guess the best parameters.
  The datasets are read one-by-one and, depending on the \tth-angle, they are either set as normalization or reflectivity data in the reduction list.
  Here is an example how the interface might look after the algorithm has finished:
  
  
  You can now scale individual datasets as described in \ref{sec:scaling}, if the stitching was not performed optimally.
  When satisfied with the result, you can save the data as described in the export section \ref{sec:export}.
  

\section{Quick start: Step-by-step standard reduction}
\label{sec:quick_start}
  For most datasets the reduction is done very similar to the fully automatized method but with more control of the user.
  Every dataset is examined by the operator to select the best extraction parameters.
  
  \subsection{Step 1: Set wavelength normalization from direct beam}
  
  \subsection{Step 2: Define a suitable background- and y-region}
  
  \subsection{Step 3: Normalize total reflection plateau}
  
  \subsection{Step 4: Add additional datasets}

  \subsection{Step 5: Refine the dataset scaling and cutting}
  \label{sec:scaling}
  
  \subsection{Step 6: Export the result}
  \label{sec:export}
  
  
\section{Examples}
  This section will give three example datasets, which you can use to try the reduction yourself and compare the result with the images in this manual.
  
\section{Common problems to be aware of}
